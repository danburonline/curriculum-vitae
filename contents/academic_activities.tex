\section{Academic Activities}

\subsection{Teaching}

\subsubsection{SAE Institute Zürich}
\coursetaught{Spring \& Fall}{Guest Lecturer}{Cloud Computing \& Software Engineering}{}{
  This course focuses on equipping students with the theoretical knowledge and practical skills to leverage modern cloud computing platforms like AWS and wrappers like Railway for software development and deployment. It covers key concepts such as microservices, containerisation with Docker, infrastructure as code (IaC), CI/CD pipelines, and database technologies. Students engage in hands-on projects and exercises using tools like GitHub Actions and object-relational mappers (ORMs) to enhance their skills in building scalable, resilient cloud-native applications.
}\\
\coursetaught{Spring \& Fall}{Guest Lecturer}{Web Specials \& Advanced Web Development}{}{
  This course covers current trends in advanced web development. The syllabus includes monorepo management with tools like Turbo, frontend development with React, 3D graphics implementation with Three.js and React Three Fiber, API-driven development, and state-of-the-art cloud deployment strategies with tools such as AWS and Cloudflare. The goal is to provide practical knowledge and techniques used by modern engineers, demonstrated through hands-on exercises and example code workflows. The course aims to make students proficient in building robust and feature-rich web applications that are production-ready.
}

\subsection{Conferences and Symposiums}
\activity{Year}{Label}{Title}{Address}{Description}\\
\activity{Year}{Label}{Title}{Address}{Description}

\subsection{Peer Reviews and Editorial Work}
\activity{Year}{Label}{Title}{Address}{Description}\\
\activity{Year}{Label}{Title}{Address}{Description}

\subsection{Mentoring}
\activity{2022+}{Industry Expert Mentor}{Advanced and Special Software Engineering}{}{I mentor and advise BSc students on their thesis and projects, providing guidance in areas such as code reviews, pair programming, design, and industry best practices. My role involves technical support as well as academic and research perspectives. I typically mentor up to three students per semester, selecting top-performing students working on particularly innovative projects. This includes projects focusing on full-body tracking and immersion in VR, user experience research for a neuroscience-informed mental health app, and technically complex projects like automated image pipelines for generative UI in e-commerce.}